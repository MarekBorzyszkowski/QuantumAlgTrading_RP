% \documentclass[conference]{IEEEtran}
% \IEEEoverridecommandlockouts
% % The preceding line is only needed to identify funding in the first footnote. If that is unneeded, please comment it out.
% \usepackage{cite}
% \usepackage{amsmath,amssymb,amsfonts}
% \usepackage{algorithmic}
% \usepackage{graphicx}
% \usepackage{textcomp}
% \usepackage{xcolor}
% \def\BibTeX{{\rm B\kern-.05em{\sc i\kern-.025em b}\kern-.08em
%     T\kern-.1667em\lower.7ex\hbox{E}\kern-.125emX}}
% \begin{document}

% \title{The use of  quantum computing in algorithmic trading}

% \author{\IEEEauthorblockN{1\textsuperscript{st} Paulina Brzcka}
% \IEEEauthorblockA{\textit{dept. of algorithms and systems modeling} \\
% \textit{Gdansk University of Technology}\\
% Gdansk, Poland \\
% s184701@student.pg.edu.pl}
% \and
% \IEEEauthorblockN{2\textsuperscript{nd} Marek Borzyszkowski}
% \IEEEauthorblockA{\textit{dept. of algorithms and systems modeling} \\
% \textit{Gdansk University of Technology}\\
% Gdansk, Poland \\
% s184701@student.pg.edu.pl}
% \and
% \IEEEauthorblockN{3\textsuperscript{rd} Wojciech Baranowski}
% \IEEEauthorblockA{\textit{dept. of algorithms and systems modeling} \\
% \textit{Gdansk University of Technology}\\
% Gdansk, Poland \\
% s184701@student.pg.edu.pl}
% }

% \maketitle



\documentclass[journal]{IEEEtran}


% *** PDF, URL AND HYPERLINK PACKAGES ***
%
%\usepackage{url}
% url.sty was written by Donald Arseneau. It provides better support for
% handling and breaking URLs. url.sty is already installed on most LaTeX
% systems. The latest version and documentation can be obtained at:
% http://www.ctan.org/pkg/url
% Basically, \url{my_url_here}.

% *** Do not adjust lengths that control margins, column widths, etc. ***
% *** Do not use packages that alter fonts (such as pslatex).         ***
% There should be no need to do such things with IEEEtran.cls V1.6 and later.
% (Unless specifically asked to do so by the journal or conference you plan
% to submit to, of course. )


% correct bad hyphenation here
\hyphenation{op-tical net-works semi-conduc-tor}


\begin{document}
%
% paper title
% Titles are generally capitalized except for words such as a, an, and, as,
% at, but, by, for, in, nor, of, on, or, the, to and up, which are usually
% not capitalized unless they are the first or last word of the title.
% Linebreaks \\ can be used within to get better formatting as desired.
% Do not put math or special symbols in the title.
\title{The Use of Quantum Computing\\ in Algorithmic Trading}
%
%
% author names and IEEE memberships
% note positions of commas and nonbreaking spaces ( ~ ) LaTeX will not break
% a structure at a ~ so this keeps an author's name from being broken across
% two lines.
% use \thanks{} to gain access to the first footnote area
% a separate \thanks must be used for each paragraph as LaTeX2e's \thanks
% was not built to handle multiple paragraphs
%
\author{Paulina~Brzecka,
        Marek~Borzyszkowski,
        Wojciech~Baranowski
        and~Piotr~Mironowicz% <-this % stops a space
\thanks{P. Brzecka, M. Borzyszkowski and W. Baranowski were with the Faculty 
of Electronics, Telecommunications and Informatics, Gdansk University of Technology, Gdansk,
Poland.}% <-this % stops a space
\thanks{P. Mironowicz, the project mentor, was with the Faculty 
of Electronics, Telecommunications and Informatics, Gdansk University of Technology, Gdansk,
Poland.}
}
% note the % following the last \IEEEmembership and also \thanks - 
% these prevent an unwanted space from occurring between the last author name
% and the end of the author line. i.e., if you had this:
% 
% \author{....lastname \thanks{...} \thanks{...} }
%                     ^------------^------------^----Do not want these spaces!
%
% a space would be appended to the last name and could cause every name on that
% line to be shifted left slightly. This is one of those "LaTeX things". For
% instance, "\textbf{A} \textbf{B}" will typeset as "A B" not "AB". To get
% "AB" then you have to do: "\textbf{A}\textbf{B}"
% \thanks is no different in this regard, so shield the last } of each \thanks
% that ends a line with a % and do not let a space in before the next \thanks.
% Spaces after \IEEEmembership other than the last one are OK (and needed) as
% you are supposed to have spaces between the names. For what it is worth,
% this is a minor point as most people would not even notice if the said evil
% space somehow managed to creep in.



% The paper headers
\markboth{Journal of \LaTeX\ Class Files,~Vol.~14, No.~8, December~2024}%
{The Use of Quantum Computing\\ in Algorithmic Trading}
% The only time the second header will appear is for the odd numbered pages
% after the title page when using the twoside option.
% 
% *** Note that you probably will NOT want to include the author's ***
% *** name in the headers of peer review papers.                   ***
% You can use \ifCLASSOPTIONpeerreview for conditional compilation here if
% you desire.




% If you want to put a publisher's ID mark on the page you can do it like
% this:
%\IEEEpubid{0000--0000/00\$00.00~\copyright~2015 IEEE}
% Remember, if you use this you must call \IEEEpubidadjcol in the second
% column for its text to clear the IEEEpubid mark.



% use for special paper notices
%\IEEEspecialpapernotice{(Invited Paper)}




% make the title area
\maketitle

% As a general rule, do not put math, special symbols or citations
% in the abstract or keywords.
\begin{abstract}
    Algorithmic trading, is an investment strategy that involves using automated trading systems to make investment decisions in financial markets. Quantum computing has the potential to enhance these strategies by processing market data and analyzing trends faster and more efficiently. This topic will explore the possibility of implementing an agent that makes investment decisions while playing the stock market using quantum computing. The agent will be tested on a quantum computer emulator or a real computer, and its effectiveness will be compared with selected algorithms that do not use quantum technologies.
\end{abstract}

% Note that keywords are not normally used for peerreview papers.
\begin{IEEEkeywords}
Algorithmic trading, investing, trading systems, financial markets, Quantum computing, market predicting.
\end{IEEEkeywords}






% For peer review papers, you can put extra information on the cover
% page as needed:
% \ifCLASSOPTIONpeerreview
% \begin{center} \bfseries EDICS Category: 3-BBND \end{center}
% \fi
%
% For peerreview papers, this IEEEtran command inserts a page break and
% creates the second title. It will be ignored for other modes.
\IEEEpeerreviewmaketitle



\section{Introduction}
% The very first letter is a 2 line initial drop letter followed
% by the rest of the first word in caps.
% 
% form to use if the first word consists of a single letter:
% \IEEEPARstart{A}{demo} file is ....
% 
% form to use if you need the single drop letter followed by
% normal text (unknown if ever used by the IEEE):
% \IEEEPARstart{A}{}demo file is ....
% 
% Some journals put the first two words in caps:
% \IEEEPARstart{T}{his demo} file is ....
% 
% Here we have the typical use of a "T" for an initial drop letter
% and "HIS" in caps to complete the first word.
\IEEEPARstart{T}{he} project will result in a draft of a scientific article describing the research conducted and its conclusions.
% You must have at least 2 lines in the paragraph with the drop letter
% (should never be an issue)


% An example of a floating figure using the graphicx package.
% Note that \label must occur AFTER (or within) \caption.
% For figures, \caption should occur after the \includegraphics.
% Note that IEEEtran v1.7 and later has special internal code that
% is designed to preserve the operation of \label within \caption
% even when the captionsoff option is in effect. However, because
% of issues like this, it may be the safest practice to put all your
% \label just after \caption rather than within \caption{}.
%
% Reminder: the "draftcls" or "draftclsnofoot", not "draft", class
% option should be used if it is desired that the figures are to be
% displayed while in draft mode.
%
%\begin{figure}[!t]
%\centering
%\includegraphics[width=2.5in]{myfigure}
% where an .eps filename suffix will be assumed under latex, 
% and a .pdf suffix will be assumed for pdflatex; or what has been declared
% via \DeclareGraphicsExtensions.
%\caption{Simulation results for the network.}
%\label{fig_sim}
%\end{figure}

% Note that the IEEE typically puts floats only at the top, even when this
% results in a large percentage of a column being occupied by floats.


% An example of a double column floating figure using two subfigures.
% (The subfig.sty package must be loaded for this to work.)
% The subfigure \label commands are set within each subfloat command,
% and the \label for the overall figure must come after \caption.
% \hfil is used as a separator to get equal spacing.
% Watch out that the combined width of all the subfigures on a 
% line do not exceed the text width or a line break will occur.
%
%\begin{figure*}[!t]
%\centering
%\subfloat[Case I]{\includegraphics[width=2.5in]{box}%
%\label{fig_first_case}}
%\hfil
%\subfloat[Case II]{\includegraphics[width=2.5in]{box}%
%\label{fig_second_case}}
%\caption{Simulation results for the network.}
%\label{fig_sim}
%\end{figure*}
%
% Note that often IEEE papers with subfigures do not employ subfigure
% captions (using the optional argument to \subfloat[]), but instead will
% reference/describe all of them (a), (b), etc., within the main caption.
% Be aware that for subfig.sty to generate the (a), (b), etc., subfigure
% labels, the optional argument to \subfloat must be present. If a
% subcaption is not desired, just leave its contents blank,
% e.g., \subfloat[].


% An example of a floating table. Note that, for IEEE style tables, the
% \caption command should come BEFORE the table and, given that table
% captions serve much like titles, are usually capitalized except for words
% such as a, an, and, as, at, but, by, for, in, nor, of, on, or, the, to
% and up, which are usually not capitalized unless they are the first or
% last word of the caption. Table text will default to \footnotesize as
% the IEEE normally uses this smaller font for tables.
% The \label must come after \caption as always.
%
%\begin{table}[!t]
%% increase table row spacing, adjust to taste
%\renewcommand{\arraystretch}{1.3}
% if using array.sty, it might be a good idea to tweak the value of
% \extrarowheight as needed to properly center the text within the cells
%\caption{An Example of a Table}
%\label{table_example}
%\centering
%% Some packages, such as MDW tools, offer better commands for making tables
%% than the plain LaTeX2e tabular which is used here.
%\begin{tabular}{|c||c|}
%\hline
%One & Two\\
%\hline
%Three & Four\\
%\hline
%\end{tabular}
%\end{table}


% Note that the IEEE does not put floats in the very first column
% - or typically anywhere on the first page for that matter. Also,
% in-text middle ("here") positioning is typically not used, but it
% is allowed and encouraged for Computer Society conferences (but
% not Computer Society journals). Most IEEE journals/conferences use
% top floats exclusively. 
% Note that, LaTeX2e, unlike IEEE journals/conferences, places
% footnotes above bottom floats. This can be corrected via the
% \fnbelowfloat command of the stfloats package.



\section{Results of the project}

\subsection{Scope of work performed and its characteristics}

\subsubsection{Collection of data sets}
Historical stock market data was collected from the top 20 companies in the WIG20 and S\&P indices, as well as historical stock market data of the indices themselves.
The data was saved in csv form for easy processing.

\subsubsection{Implementation of classical algorithms}
With the help of available sources, implementations of algorithms based on the classical approach to computation were added. 
The selected algorithms were:
\begin{itemize}
	\item PCA,
	\item SVM.
\end{itemize}
\subsubsection{Implementation of quantum algorithms}
With the help of available sources, implementations of algorithms based on the quantum computing approach were added. 
The selected algorithms were:
\begin{itemize}
	\item QPCA,
	\item QSVM.
\end{itemize}
\subsubsection{Conducting tests}
On the basis of selected data sets (WIG20/S\&P), the prediction quality of selected classical as well as quantum algorithms was checked.

\subsubsection{Start of preliminary work on the article}
A preliminary outline of the article was created, and the necessary bibliography was collected.

\subsection{Characteristics of teamwork}
During the research work, the following tools were used to exchange ideas and created artifacts:
\begin{itemize}
	\item discord to meet in a group as well as with a mentor, 
	\item github to hold artifacts produced while working on the project,
	\item \LaTeX ~to write documents.
\end{itemize}
\subsection{Achieved results}
This semester's research project involved the implementation of selected algorithms for predicting continuous values from a data set.
The selected algorithms were:
\begin{itemize}
	\item PCA,
	\item SVM,
	\item QPCA,
	\item QSVM.
\end{itemize}
For each algorithm, wide-ranging tests were conducted on data covering stock market data from previous years. 
Based on the tests, it was possible to create graphs showing the performance of each algorithm, as well as collect statistical data on the quality of each algorithm's prediction.
\subsection{Discrepancies and changes in project implementation}
None.
\subsection{Provisions}
Performing more tests and completing the article.

% that's all folks
\end{document}