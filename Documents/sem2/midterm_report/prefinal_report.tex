\documentclass[polish,envcountsect,10pt]{article}

   	\usepackage[T1]{fontenc}
   	\usepackage{polski}
    \usepackage{babel}
    \usepackage{subfigure}
	\usepackage{graphicx}
	\usepackage{geometry}
	\usepackage{listings}
	\usepackage{float}
	\usepackage{graphicx}

\title{Raport przejściowy}
\author{inż. Paulina Brzęcka 184701 \and inż. Marek Borzyszkowski 184266 \and inż. Wojciech Baranowski 184574}
\date{\today}
\begin{document}

\maketitle
\tableofcontents
\newpage

\section{Projekt badawczy}

\subsection{Tytuł}

Wykorzystanie obliczeń kwantowych w algorithmic trading.

\subsection{Zleceniodawca i Opiekun}

Zleceniodawcą i opiekunem projektu jest dr inż. Piotr Mironowicz.

\subsection{Uczelnia i wydział}

Politechnika Gdańska - Wydział Elektroniki, Telekomunikacji i Informatyki.

\section{Rezultaty projektu}

\subsection{Założenia początkowe i krótki opis projektu}
Algorithmic trading, czyli handel algorytmiczny, to strategia inwestycyjna polegająca na wykorzystaniu zautomatyzowanych systemów handlowych do podejmowania decyzji inwestycyjnych na rynkach finansowych. Obliczenia kwantowe mają potencjał wzmocnienia tych strategii poprzez szybsze i bardziej efektywne przetwarzanie danych rynkowych oraz analizę trendów. W ramach tego tematu zostanie zbadana możliwość zaimplementowania agenta podejmującego decyzje inwestycyjne podczas gry na giełdzie, wykorzystując obliczenia kwantowe. Agent będzie testowany na emulatorze komputera kwantowego lub rzeczywistym komputerze, a jego skuteczność będzie porównywana z wybranymi algorytmami niekorzystającymi z technologii kwantowych. Efektem projektu będzie szkic artykułu naukowego opisującego przeprowadzone badania i wnioski z nich płynące.

\subsection{Zakres wykonanych prac i ich charakterystyka}

\subsubsection{Zebranie zestawów danych}
Zebrano historyczne dane giełdowe z top 20 spółek z indeksów WIG20 i S\&P, jak i historyczne dane giełdowe samych indeksów.
Dane te zapisano w formie csv w celu łatwego ich przetwarzania.

\subsubsection{Implementacja algorytmów klasycznych}
Z pomocą dostępnych źródeł dodkonano implementacji algorytmów bazujących na klasycznym podejściu do obliczeń. 
Wybranymi algorytmami były:
\begin{itemize}
	\item PCA,
	\item SVM.
\end{itemize}
\subsubsection{Implementacja algorytmów kwantowych}
Z pomocą dostępnych źródeł dodkonano implementacji algorytmów bazujących na kwantowym podejściu do obliczeń. 
Wybranymi algorytmami były:
\begin{itemize}
	\item QPCA,
	\item QSVM.
\end{itemize}
\subsubsection{Przeprowadzenie testów}
Na podstawie wybranych zestawów danych (WIG20/S\&P), sprawdzono jakość predykcji wybranych algorytmów klasycznych, jak i kwantowych.

\subsubsection{Rozpoczęcie wstępnych prac nad artykułem}
Stworzono wstępny zarys artykułu, zebrano potrzebną bibliografię.

\subsection{Charakterystyka pracy zespołowej}
Podczas prac badawczych korzystano z następujących narzędzi do wymiany myśli i stworzonych artefaktów:
\begin{itemize}
	\item discord do spotkań w grupie jak i z opiekunem, 
	\item github do trzymania artefaktów wytworzonych w czasie pracy nad projektem,
	\item \LaTeX ~do pisania dokumentów.
\end{itemize}
\subsection{Osiągnięte wyniki}
W ramach projektu badawczego w tym semestrze dokonano implementacji wybranych algorytmów służących do przewidywania wartości ciągłych na podstawie zestawu danych.
Wybranymi algorytmami były:
\begin{itemize}
	\item PCA,
	\item SVM,
	\item QPCA,
	\item QSVM.
\end{itemize}
Dla każdego z algorytmów przeprowadzono testy o szerokim zakresie na danych pokrywające dane giełdowe z ubiegłych lat. 
Na podstawie testów udało się stworzyć wykresy ukazujące działanie każdego z algorytmów, jak i zebrano dane statystyczne dotyczące jakości predykcji każdego z algorytmów.
\subsection{Rozbieżności i zmiany w realizacji projektu}
Brak.
\subsection{Postanowienia}
Wykonanie większej liczby testów oraz skończenie artykułu.

\end{document}