\documentclass[polish,envcountsect,10pt]{article}
\usepackage[T1]{fontenc}
   	\usepackage{polski}
    \usepackage{babel}
	\usepackage{subfigure}
	\usepackage{graphicx}
	\usepackage{geometry}
	\usepackage{listings}
	\usepackage{float}
 \usepackage{graphicx}


	%\usepackage[hidelinks]{hyperref}

\title{Wstępny pomysł na testowanie algorytmów}
\author{inż. Paulina Brzęcka 184701 \and inż. Marek Borzyszkowski 184266 \and inż. Wojciech Baranowski 184574}
\date{\today}
\begin{document}

\maketitle
\tableofcontents
\newpage

\section{Wstęp}

Podczas pracy nad algorytmami grającymi na giełdzie, kwantowymi lub klasycznymi, przydatne będą.
\begin{itemize}
    \item standaryzacja uruchamiania algorytmów,
    \item standaryzacja procesu testowania algorytmów,
    \item standaryzacja danych udostępnianych przez algorytm.
\end{itemize}

Z powodu powyższych powstał ten dokument, dający wstępny zarys jak może ( lub nawet będzie ) wyglądać każdy z powyższych elementów.

\section{Dane}
Przed algorytmem warto zdefiniować na jakich danych będą testowane algorytmy.
Pozwoli to uprościć definicję struktury klasy z algorytmem oraz wskazać kierunek przy decyzjach implementacyjnych.
Z założenia dane wykorzystywane to będą:
\begin{enumerate}
    \item Seria odczytów indeksów giełdowych, takich jak S\&P500, WIG20 etc. Odczyty będą w odstępach dziennych.
    \item Odczyty poszczególnych aktywów wchodzących w dany indeks ( zazwyczaj akcje spółek ). Odczyt jw. Dla danego indeksu zestaw odczytów poszczególnych aktywów będzie przechowywany w tablicy dwuwymiarowej.
    \item Modele poszczególnych algorytmów, które wymagają trenowania przed ich użyciem. Jest to szczególnie ważne przy dalszej pracy, gdy potrzebne będzie wrócenie do modelu wcześniej przetrenowanego w celu porównania wyników.
    \item Wyniki poszczególnych algorytmów w postaci pliku CSV zawierającego serię dat, liczby danych aktywów, łączna wartość aktywów.
    \item Metadane o algorytmie w postaci pliku, takie jak czas działania na zadanej serii danych, czas trenowania modelu.   
\end{enumerate}
\section{Algorytm}

Podstawą algorytmów będzie \texttt{TradingAlgorytm}. \texttt{TradingAlgorytm}  będzie klasą abstrakcyjną, po której dziedziczyć będą inne algorytmy, klasyczne i kwantowe.
Metody abstrakcyjne jakie będzie posiadać klasa \texttt{TradingAlgorytm}:
\begin{itemize}
    \item \texttt{train}
    \item\texttt{fit}
    \item \texttt{history}
    \item \texttt{save}
    \item \texttt{load}
\end{itemize}


\subsection{\texttt{train}}
Input: Tablica dwuwymiarowa, dla danego przedziału czasu ( w dniach ) jaka jest wartość danej akcji.\\
Output: algorytm/brak?\\
Metoda ta ma za zadanie wytrenować algorytm na wybranym zestawie akcji. 
Ważne aby dane użyte do trenowania danego algorytmu były danymi wcześniejszymi, niż dane później wykorzystywanymi do ewaluacji danego algorytmu.
Najlepiej jakby to były dane 4-5 letnie przed datą wybraną jako rozpoczęcie predykcji.
Dla algorytmów niewymagających uprzedniego uczenia ( patrz MACD ), powinno zwracać nic?
\subsection{\texttt{fit}}
Input: Tablica dwuwymiarowa, dla danego przedziału czasu ( w dniach ) jaka jest wartość danej akcji?\\
Output: Wyniki modelu w postaci tabeli, wyżej opisane w sekcji \emph{Dane}\\
Metoda robiąca predykcje na zadanej serii krotek aktywów.
Ważne, dane muszą być inne niż dane do trenowania.
\subsection{\texttt{history}}
Input: brak?\\
Output: historia procesu uczenia danego algorytmu?
\subsection{\texttt{save}}
Input: brak\\
Output: brak\\
Metoda zapisująca algorytm ( i może historię? ) do folderu. 
\subsection{\texttt{load}}
Input: Ścieżka dojścia do folderu z danym algorytmem.\\
Output: Brak?\\
Metoda ładująca algorytm do danej instancji algorytmu?

\end{document}